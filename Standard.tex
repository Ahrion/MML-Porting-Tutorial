\section{Spezialbefehle, Standardbefehle und Makros}

In diesem Kapitel werden einige Spezial- und Standardbefehle vorgestellt.
Eine vollständige Liste aller Befehle kann unter: \\ \textit{AddmusicK/readme\_files/syntax\_reference.html} eingesehen werden.

\subsection{Spezialbefehle}

Spezialbefehle erkennt man daran, dass sie mit einem \# beginnen (einizge Ausnahmen sind Kanalnummern). Alle Spezialbefehle müssen sich außerhalb der Kanäle befinden, einige müssen zwingend über den Kanälen oder in einer gewissen Reihenfolge platziert werden.


\subsubsection*{\#path \dq Pfadname\dq{}}

Falls Custom Samples verwendet werden sollen, gibt \#path den Pfad an, wo diese liegen (müssen sich in einem Unterverzeichnis von samples/ befinden).

\subsubsection*{\#samples\{\}}

Verwendete Custom Samples werden in \#samples\{\} gelistet. Neben den Samples muss sich zusätzlich eine Samplegroup bestimmt werden.

\medskip

\lstinputlisting[framexleftmargin=8mm, frame=shadowbox, rulesepcolor=\color{blue}, numbers=left, firstline=1, lastline=6]{codes/Spezial.txt}

\medskip

Anstelle von \#path kann auch der Pfad der Samples in den Namen angegeben werden.

\subsubsection*{\#samples\{\}}

In \#instruments\{\} können ADSR, Gain und Pitch eines SMW Samples, Custom Samples, oder Noise (Ruaschen, wird benutzt um 8 Bit Percussions zu simulieren) definiert werden. Neu definierte Instrumente werden durchnummeriert, angefangen mit @30 und mit @ aufgerufen.

\medskip

\lstinputlisting[framexleftmargin=8mm, frame=shadowbox, rulesepcolor=\color{blue}, numbers=left, firstline=7, lastline=12]{codes/Spezial.txt}

\medskip

Für weitere Informationen zu \#samples\{\}, \#samplegroups und \#instruments\{\} siehe Kapitel \ref{sec:instrumente}.

\subsubsection*{\#spc}

Mit \#spc können Informationen zum Song hinterlegt werden, die in Lunar Magic und SPC Playern angezeigt werden.

\medskip

\lstinputlisting[framexleftmargin=8mm, frame=shadowbox, rulesepcolor=\color{blue}, numbers=left, firstline=13, lastline=20]{codes/Spezial.txt}

\medskip

Es können beliebig viele dieser Einträge benutzt werden.

\subsubsection*{\#halvetempo}

Halbiert das Tempo, Notenlängen und Hexbefehle, die eine Dauer verwenden. Nützlich, um Slowdowns zu umgehen, da diese maßgeblich durch zu hohes Tempo auftreten.

\subsubsection*{\#option}

Mit \#option, gefolgt von einem Schlüsselwort, können folgende Einstellungen vorgenommen werden:

\begin{table}[htbp]
	\begin{tabularx}{\textwidth}{|l|X|}
		\hline
		Parameter & Beschreibung\\
		\hline
		tempoimmunity & Das Tempo des Songs wird nicht erhöht, falls der SMW Timer unter 100 Sekunden ist.\\
		\hline
		dividetempo & Wie \#halvetempo, es kann aber ein anderer Divisor als 2 ausgewählt werden.\\
		\hline
		smwvtable & Der Song benutzt die SMW Lautstärketabelle anstatt der standard N-SPC Tabelle\\
		\hline
		noloop & Der Song wird nur ein einziges mal abgespielt und nicht geloopt.\\
		\hline
	\end{tabularx}
\end{table}

Beispiel: \#option dividetempo 3


\subsubsection*{\#define, \#undef, \#ifdef, \#ifndef, \#if, \#endif, \#error}

\subsection{Standardbefehle}
\subsubsection{Loops}
\subsubsection{Intros}
\subsubsection{Makros}
\subsubsection{Remote Codes}